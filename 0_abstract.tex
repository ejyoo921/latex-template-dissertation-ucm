
\par
Settling marine aggregates are essential in transporting dissolved carbon dioxide from the ocean surface to the deep sea. While sinking, they may accumulate in thin layers where density stratifications are present, becoming nutrient hotspots for bacteria and animal activities. We thus simulate settling aggregates both in ambient homogeneous and density-stratified fluid to study their dynamics. 
% We consider both ambient homogeneous and density-stratified fluid. 
We model a marine aggregate as a fractal collection of cubes. 
% The homogeneous flow around the aggregate is computed in the limit of zero Reynolds number using a boundary integral method. 
In a homogeneous fluid, 
the flow is computed in the Stokes regime using a boundary integral method.
We then focus on analyzing the forces acting on the aggregate in the presence of various background flows. 
In the presence of a stratified ambient, a term involving a volume integral is added to the boundary integral formulation. We use the fast multipole method 
% with a modified Laplace's kernel 
for rapid computation in three dimensions. We couple the velocity with the advection-diffusion equation to track the temperature or salt concentration in time. We use this method to quantify how the presence of stratification affects the settling speed and time of the aggregates.
\par
We also present the study of complex fluids with higher-order rheology. We introduce rate-dependent flows having non-Newtonian behaviors. 
We then focus on a flow of granular materials, which is the second most commonly used in industry after water.
We are interested in a rheological model for granular material that is second-order in strain rate and exhibits yield stress.
We implement a viscoplastic model within the Adaptive Mesh Refinement framework to solve for incompressible flows.
We also formulate a regularization of the yield stress terms to account for viscosity divergence at low strain rates. 
To incorporate a higher-order strain tensor with the incompressible solver numerically, we may use an explicit method for time integrations. In addition to the default solver, the explicit Euler method, we added a two-stages Runge-Kutta scheme.