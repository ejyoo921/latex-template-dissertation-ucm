
\par
Settling marine aggregates is essential in transporting dissolved carbon dioxide (CO$_2$) from the ocean surface to the deep sea level. While sinking, they accumulate in thin layers where density stratifications are present, becoming nutrient hotspots for bacteria and animal activities. For this reason, we simulate settling aggregates in a fluid to study their behavior. We consider both ambient homogeneous and density-stratified fluid. We assemble fractal aggregates as collections of cubes to model a marine aggregate. The homogeneous flow around the aggregate is computed in the limit of zero Reynolds number using a boundary integral method. 
In a homogeneous fluid, we focus on analyzing forces acting on the aggregate depending on the background flow types. 
A term involving a volume integral is added to the velocity solution to have variable fluid density. We use the fast multipole method with a modified Laplace's kernel for rapid computation in three dimensions. We couple the velocity with the advection-diffusion equation to track the heat density or salt concentration in time. We use this method to quantify how the presence of stratification affects the settling speed and time of the aggregates.
\par
In addition, we study complex fluids with higher-order rheology. 
We particularly focus on a flow of 
a specific class of complex fluid, granular materials, which is the second most commonly found on earth after water. 
We are interested in a rheological model for granular material, that is second-order in strain rate and exhibits yield stress.
We implement a viscoplastic model within the Adaptive Mesh Refinement (AMR) framework to solve for incompressible flows.
We also formulate a regularization of the yield stress terms to account for viscosity divergence at low strain rates.