\par
In this chapter, I report the internship research I began at LBNL in the summer of 2022 that I have continued to work on as part of my doctoral studies.
\par
% We particularly study the granular materials for non-Newtonian flow behavior. 



\section{Flow equations}
Assuming pressure-independent, we incorporate the varying viscosity into the incompressible Navier-Stokes equations, which are introduced in~(\ref{eq_conserv_mass}) and~(\ref{eq_momentum_NS}), recalling here,
\begin{align}
  \nabla \cdot \vec{u} = 0 
  \nonumber
  % \label{eq_div_free} 
  \\
  \frac{\partial \vec{u}}{\partial t} + \vec{u}\cdot \nabla \vec{u}
  = \frac{1}{\rho}
  \left(
  - \nabla P 
      + \nabla \cdot   \bm{\tau} 
      +  \rho  \vec{g} 
      \right),
      \nonumber
    % \label{eq_NS_ch4}
\end{align}
to obtain the flow velocity, $\vec{u}$.
\begin{comment}
To explain terminology, we write the stress tensor $\bar{\bar{\sigma}}$ explicitly, 
\begin{equation}
  \bar{\bar{\sigma}} = 
  \begin{bmatrix}
    \sigma_{xx} & \sigma_{xy} & \sigma_{xz} 
    \\
    \sigma_{xy} & \sigma_{yy} & \sigma_{zy} 
    \\
    \sigma_{xz} & \sigma_{zy} & \sigma_{zz}.
  \end{bmatrix}
  \label{eq_cauchy_mx}
\end{equation} 
Here, we define the first normal stress difference $N_1$
\begin{equation}
  N_1 = \sigma_{xx} - \sigma_{yy}
\end{equation}
and the second normal stress difference $N_2$,
\begin{equation}
  N_2 = \sigma_{yy} - \sigma_{zz}.
\end{equation}
the deviatoric stress tensor requires a higher-order relationship with the strain rate tensor $\bm{D}$. 
\end{comment}
We include a viscous stress tensor of the form,
\begin{equation}
   \boldsymbol{\tau} = 
   % 2 \nu_1  \bm{D} + 2 \nu_2  \bm{D}^2 ,
    \nu_0  \bm{A}_1 +  \nu_1  \bm{A}_1^2 + \nu_2 \bm{A}_2,
   \label{eq_CN_tau}
\end{equation}
where $\nu_0$ is constant in shear flow, and $\nu_1$ and $\nu_2$ are a function of the shear rate $\dot{\gamma}$.
The second-order form of the deviatoric stress tensor, equation~(\ref{eq_CN_tau}), was introduced by Colemann and Noll~\cite{coleman_approximation_1960}.
The term $\bm{A_i}$ is the Rivlin-Ericksen tensor, defined as
\begin{equation}
   {\bm A_1}  = \nabla \vec{u} +  \left( \nabla \vec{u} \right)^T = 2 \bm{D} 
   \label{eq_A1}
\end{equation}
and 
\begin{equation}
   \boldsymbol{A}_2
   % =\frac{D}{D t} \boldsymbol{A}_1+\boldsymbol{A}_1 \nabla \vec{u}+ \left(\nabla \vec{u} \right)^T \boldsymbol{A}_1
   =\frac{\partial \boldsymbol{A}_1}{\partial t} + \vec{u} \cdot \nabla \boldsymbol{A}_1+\boldsymbol{A}_1 \nabla \vec{u}+ \left(\nabla \vec{u} \right)^T \boldsymbol{A}_1.
   \label{eq_A2}
\end{equation}
Note that the general form of the Rivlin-Ericksen tensor can be extended to order $n$,  
\begin{equation}
  \boldsymbol{A}_{n+1}
  =\frac{\partial \boldsymbol{A}_{n}}{\partial t} + \vec{u} \cdot \nabla \boldsymbol{A}_n+\boldsymbol{A}_n \nabla \vec{u}+ \left(\nabla \vec{u} \right)^T \boldsymbol{A}_n.
\end{equation}
\par
In general, many researchers do not include the effect of $\mathcal{O}\left( \bm{D}^2 \right)$ for examining a complex fluid. For polymer solutions, the most studied in non-Newtonian fluids, it is well-known that the effect of the $\mathcal{O}\left( \bm{D}^2 \right)$ term is small enough to ignore, compared to  $\mathcal{O}\left( \bm{D} \right)$~\cite{bird_dynamics_1987}. 
However, there are potentially important characteristics that are neglected when we consider the stress only linear in $\boldsymbol{D}$ for other materials, including granular material. We will explore both first- and second-order rheologies, depending on the materials we consider. 
In the following four sections, we discuss the viscous tensor term $\bm \tau$ for each type of complex fluid. 

\section{Rate-dependent viscosity}
In general, most non-Newtonian behaviors of complex fluids can be captured via shear rate dependency. 
We can describe rate-dependent viscosity with the following power law~\cite{herschel_konsistenzmessungen_1926}. We then have the viscous stress tensor, 
\begin{equation}
  {\bm \tau}
  =2^n \tilde{\mu} \dot{\gamma} ^{n-1} {\bm D},
  \label{eq_HS_tau}
  \end{equation}
where $\tilde{\mu}$ is constant viscosity under zero shear rate and $n$ is the flow behavior index that can be determined experimentally depending on the materials.
% Note that we return to the Newtonian fluid viscosity when $n = 1$, as introduced in equation (\ref{eq_stress_tensor}). 
% As we described in Figure~\ref{fig_rheology}, we see a shear-thinning fluid, also called pseudoplastic fluid, if 
\begin{figure}[ht]
  \begin{center}
    \includegraphics[scale=0.19]{figures/fig_appr_viscosity.pdf}
    \end{center}
  \caption{Relationship between apparent viscosity and shear rate. The plot is drawn based on~\cite{irgens_rheology_2014}. The value $\eta_{\infty}$ is the asymptotic viscosity at high shear rate.}
  \label{fig_appr_viscosity}
\end{figure}
The relationship between rate-dependent viscosity and shear rate can be addressed with the \textit{apparent viscosity} function, $\eta$,
% To understand the relationship between rate-dependent viscosity and shear rate better, we introduce the function called \textit{apparent viscosity}, $\eta$,
\begin{equation}
  \eta(\dot{\gamma})
    =2^{n-1} \tilde{\mu} \dot{\gamma} ^{n-1}.
  \label{eq_appr_viscosity}
\end{equation}
We show the variation of apparent viscosity depending on the shear rate in Figure~\ref{fig_appr_viscosity}.
The black dashed line is when $n = 1$ in equation~(\ref{eq_appr_viscosity}); we find the Newtonian fluid viscosity $\tilde{\mu}$ as introduced in equation (\ref{eq_stress_tensor}). 
% Note that $\eta_{\infty}$ is the asymptotic viscosity value at high shear rate.
Furthermore, the purple and green lines in Figure~\ref{fig_appr_viscosity} imply that the shear-thinning and thickening behaviors can be found when $n < 1$ and $n>1$, respectively. 
We will discuss more of Bingham fluid and yield stress in the following section. 

\section{Yield stress flow}
% The main characteristic of rate-dependent fluids is yield stress. 
As we have shown in Figure~\ref{fig_rheology} in Chapter 1.3, Bingham fluid is a particular case of the rate-dependant viscosity fluid. 
Bingham fluid behaves as a solid, like plastic, under stress less than the yield stress, and begins flowing like a fluid with varying viscosity. 
This is why it is classified as a \textit{viscoplastic} fluid.
The rheological equations describing Bingham fluid~\cite{bingham_investigation_1917} with yield stress $\tau_0$ are,
\begin{align}
  \left\{\begin{matrix}
 {\bm D} =0, & \text { if } |{\bm \tau} | <\tau_0 \\
  {\bm \tau}  = 
\left(2 \tilde{\mu} + \frac{\tau_0}{ \dot{\gamma}}\right) {\bm D} , & \text { if } |{\bm \tau}|\geq \tau_0,
\end{matrix}\right.
\label{eq_Bingham}
  \end{align}
and its corresponding apparent viscosity is
\begin{equation}
  \eta=\tilde{\mu} +\frac{\tau_0}{2 \dot{\gamma}}.
  \end{equation}
  To illustrate more general yield stress fluid,~\cite{sverdrup_highly_2018} proposed the constitutive equations by combining equations~(\ref{eq_HS_tau}) and~(\ref{eq_Bingham}),
  \begin{equation}
    \left\{\begin{matrix}
  {\bm D }=0, 
    & \text { if } |{\bm \tau}| < \tau_0 
    \\
    {\bm \tau}  =
    \left(2^n \tilde{\mu} \dot{\gamma}^{n-1} + \frac{\tau_0}{ \dot{\gamma}}\right) {\bm D}, 
    & \text { if } |{\bm \tau}|  \geq \tau_0,
  \end{matrix}\right. 
  \label{eq_HS_B}
   \end{equation}
When $n = 1$, we recover the Bingham fluid.
% where $k$ is the consistency index, which is also determined by experiments along with the behavior index $n$, by measuring the fluid viscosity under the various shear rates, fitting the power law.
% A few examples of these indices $n$ and $k$ can be found in~\cite{irgens_rheology_2014}.
\subsubsection{Viscosity regularization}
As it is implied in equation~(\ref{eq_HS_B}), we obtain the yield stress of a fluid as $|\bm{\tau}| \rightarrow \tau_0$. 
When we solve for $\tau_0$ from equation~(\ref{eq_HS_B}), we see that 
\begin{equation}
2^n \tilde{\mu} \dot{\gamma}^{n} + \tau_0
\rightarrow \tau_0,
\end{equation}
due to $|\bm{D}| = \dot{\gamma}$, and thus, we should have 
\begin{equation}
  \dot{\gamma} \rightarrow 0.
\end{equation}
For this computation, we use the Papanastasiou regularization method~\cite{papanastasiou_flows_1987} since it is implemented in the AMReX-incflo. 
\par
By introducing a small parameter, denoted as $\varepsilon$, we can regularize the singularity of $\dot{\gamma}$ as following,
\[
  \frac{1}{\dot{\gamma}} \rightarrow \frac{1-e^{-\dot{\gamma} / \varepsilon}}{\dot{\gamma}}  
\]
for $\dot{\gamma}/\varepsilon \gg 1$. Otherwise, we simply use $1/\varepsilon$. 
The detailed mathematics and analysis can be found in~\cite{sverdrup_highly_2018}.
\section{Second-order strain rate rheolgy}
We now consider the second-order strain rate term as introduced in~(\ref{eq_A2}), as well as the first-order term~(\ref{eq_A1}).
Examples included the suspension flow from Guazelli and the rod-climbing effect. The coefficients $\nu_0$ in the general form of the Rivlin-Ericksen tensor can be obtained by some formulae. 

We hereby write the second-order strain rate to understand what we need to implement numerically in the AMReX-incflo. 


We consider
\[
\vec{u}= (u, \ v, \ w)
\]
\[
\nabla \vec{u}
= 
\begin{bmatrix}
u_x & u_y & u_z
\\
v_x & v_y & v_z
\\
w_x & w_y & w_z
\end{bmatrix}
\]
We want to compute
\[
{\bm D} = \nabla \vec{u} + \left( \nabla \vec{u} \right)^T
\]


Let's see what $\bm{D}^2 = \left[D^2_1, \ D^2_2, \ D^2_3 \right]$ looks like first: 
\\
First column 
\[
D^2_1 = 
\begin{bmatrix}
 \left(u_y+v_x\right)^2+\left(u_z+w_x\right)^2+4 u_x^2 
 \\
  \left(u_z+w_x\right) \left(v_z+w_y\right)+2 \left(u_x+v_y\right) \left(u_y+v_x\right)
  \\
  \left(u_y+v_x\right) \left(v_z+w_y\right)+2 \left(u_x+w_z\right) \left(u_z+w_x\right)
\end{bmatrix}
\]
Second column
\[
D^2_2 = 
\begin{bmatrix}
 \left(u_z+w_x\right) \left(v_z+w_y\right)+2 \left(u_x+v_y\right) \left(u_y+v_x\right)
 \\
 \left(u_y+v_x\right)^2+\left(v_z+w_y\right)^2+4 v_y^2
 \\
 \left(u_y+v_x\right) \left(u_z+w_x\right)+2 \left(v_y+w_z\right) \left(v_z+w_y\right)
\end{bmatrix}
\]
Third column
\[
D^2_3=
\begin{bmatrix}
 \left(u_y+v_x\right) \left(v_z+w_y\right)+2 \left(u_x+w_z\right) \left(u_z+w_x\right)
 \\
 \left(u_y+v_x\right) \left(u_z+w_x\right)+2 \left(v_y+w_z\right) \left(v_z+w_y\right)
 \\
 \left(u_z+w_x\right)^2+\left(v_z+w_y\right)^2+4 w_z^2
\end{bmatrix}
\]
\section{Granular rheology}
We now model a new stress tensor $\bm \tau$ with a higher-order strain rate that can describe non-isotropic material flow properties,
\begin{align}
  \bar{\bar{\sigma}}
    = -P \bar{\bar{I}}  + \bm{\tau}
    =  -P \bar{\bar{I}}  
    + \mu_1(\dot{\gamma}) \mathcal{O}({\bm D})
    + \mu_2(\dot{\gamma}) \mathcal{O}({\bm D^2}).
  \end{align}
In this section, we focus on the methodology to compute the viscosity $\mu_i ({\dot{\gamma}})$ ($i = 1,2$) under a simple shear flow. 
% A new apparent viscosity computation using the well-known $\mu(I)$ relation can also be found [{\color{blue}REFERENCE}]. 

To describe a non-isotropic material, we consider a deviatoric stress tensor of the form~(\ref{eq_2ndOrder_tau}), 
% \begin{align*}
% \bar{\bar{\sigma}} - P {\bm I} \equiv
%   {\bm {\bm \tau}}
%   =  \ \mu_1 {\bm D} 
%   + \mu_2  \left[ {\bm D}^2  - \frac{\text{tr}\left({\bm D}^2\right)}{3}{\bm I} \right]
%   + \eta_3  \left[ {\bm D}{\bm W} - {\bm W}{\bm D} \right]
%   \nonumber \\
%   + \frac{\kappa_1}{\dot{\gamma }^2} \bm D 
%   + \frac{\kappa_2}{\dot{\gamma }^2}  \left[ {\bm D}^2  
%   - \frac{\text{tr}\left({\bm D}^2\right)}{3}{\bm I} \right],
% \end{align*}
\begin{equation}
  {\bm {\bm \tau}}
  = \mu_1(\dot{\gamma}) {\bm D}
  \ +  \ 
 \mu_2 (\dot{\gamma})
  \left[ {\bm D}^2  - \frac{\text{tr}\left({\bm D}^2\right)}{3}{\bm I} \right],
  % \ + \
  % \left( \mu_3 \dot{\gamma}^2 \right)
  % \frac{1}{\dot{\gamma}^2}
  %   \left[ {\bm D}{\bm W} - {\bm W}{\bm D} \right]
\label{eq_2ndOrder_tau}
\end{equation}
where 
\begin{equation}
  \mu_1 (\dot{\gamma})
   = \left( \eta_1 \dot{\gamma}+ \kappa_1 \right) \frac{1}{\dot{\gamma}},
\label{eq_mu1_main}
\end{equation}
and 
\begin{equation}
  \mu_2 (\dot{\gamma}) = 
  \left( \eta_2  \dot{\gamma}^2
  +  \kappa_2 
  \right) \frac{1}{\dot{\gamma}^2},
  \label{eq_mu2_main}
\end{equation}
which is derived under two conditions:
(1) the flow motion is simple with a constant stretch history by neglecting deformation history, and (2) the flow is isochoric, having uniform properties along streamlines and tr$(D) = 0$. 
% In particular, we see \cite{srivastava_viscometric_2021}

Note that the magnitude of strain rate, $\dot{\gamma}$, can be computed by applying the scaled Frobenius norm for a second-order tensor, 
\[
  \dot{\gamma}  = |\bm{D}| = \sqrt{\frac{1}{2}
    \text{tr}\left(\bm{D} \bm{D}^{T} \right)}.
\]
In general, the Frobenius norm states $\bm{D}^H$ instead of $\bm{D}^T$. Since we consider real-valued tensors.
% it is valid to have $\bm{D}^H = \bm{D}^T$.
\par
The flow functions, $\eta_i$ and $\kappa_i$ for $i = 1,2$, have the following depending on total stress: each $\eta_i(\dot{\gamma}, p)$ is (shear) rate-dependent and $\kappa_i (p)$ is (shear) rate-independent. 
% Along with the shear effect from $\eta_1$, we can observe the second normal-stress difference in shear flows from $\eta_2$.
% The rate-independent terms, $\kappa_1$ and $\kappa_2$, allow us to find yield stress. 
We explain the details of these two types of functions in the following sub-section.

\subsection{$\mu (I)$ rheology}
The key to modeling the $\eta_i$ and $\kappa_i$ terms is the well-known $\mu(I)$ relationship developed by~\cite{jop_constitutive_2006}.
Here, $I$ is the \textit{inertial number},
\begin{equation}
  I =  \frac{\dot{\gamma} d }{\sqrt{P/\rho_p}},
  \label{eq_inertialI}
\end{equation}
where $d$ and $\rho_p$ are the average particle diameter and density of a given granular material.
This dimensionless quantity describes the ratio of the average static force to the inertial force between granular particles~\cite{jop_constitutive_2006}, interpreted the inertial number as the ratio between a macroscopic deformation and an inertial timescale. 
\par
To understand the granular flow regimes depending on the inertial number. We consider an hourglass example. 
\begin{figure}[ht]
  \begin{center}
    \includegraphics[scale=0.15]{figures/fig_hourglass.pdf}
    \end{center}
  \caption{Schematics of granular flow with an hourglass example.}
  \label{fig_hourglass}
\end{figure}
As we can see in Figure~\ref{fig_hourglass}, three different states can co-exist in granular materials. 
In the top part of the hourglass, filled with sand, the particles seem not to move and resemble a solid. 
In the middle nozzle of the hourglass, we can observe the sands passing through the nozzle, flowing like a liquid.
Meanwhile, the sand landing on the bottom part of the hourglass is forming a cone shape. When we look very closely at the top of the cone, we can see that the sand is falling and colliding; behaving like a gas.
\par
In general, these three regimes can be categorized by the inertial number $I$. The solid-like state appears when $I$ is small. As $I$ increases, the granular material deformation occurs rapidly, following as we see in the middle of the hourglass. The collisional flow, acting like a gas, can be observed for a large $I$ number. 
\par
By applying this $\mu(I)$ rheology, Srivastava~\cite{srivastava_viscometric_2021} devleoped a callibration to obtain the coefficients $\eta_i$ and $\kappa_i$.
For the first-order strain rate viscosity term, equation~(\ref{eq_mu1_main}), we have 
\begin{equation}
  \mu_1(I) = \mu_1^0 + A_1{ I}^{ \alpha_1} =  \frac{(\eta_1 \dot{\gamma} + \kappa_1)}{P},\
\label{eq_muI1}
\end{equation}
where $\mu_1^0, A_1,$ and $\alpha_1$ are fitting parameters, depending on the matrials. 
By substituting the inertial number $I$, equation~(\ref{eq_inertialI}), into equation~(\ref{eq_muI1}), we get
\begin{equation}
  \mu_1(I) =
     \mu_1^0 + A_1 {\left(  \frac{\dot{\gamma} a }{\sqrt{P/\rho}}\right) }^{ \alpha_1} =  \frac{(\eta_1 \dot{\gamma} + \kappa_1)}{P}.
\end{equation}
We then find $\mu_1 (\dot{\gamma}, P)$ and $\kappa_1(p)$ as
\begin{equation}
    \mu_1  (\dot{\gamma}, P)= 
    \biggl( A_1 {\left(   d  \sqrt{\rho} \right) }^{ \alpha_1}\biggr) 
     \dot{\gamma}^{\alpha-1} P^{1-\alpha/2}
\label{eq_eta1}
\end{equation}
\begin{equation}
    \kappa_1(P) = \mu_1^0 P
\label{eq_kappa1}
\end{equation}
\par
Similarly, we can find the coefficients for the quadratic in ${\bm D}$ terms, having the $\mu(I)$ relationship as follows:
\begin{equation}
    \mu_2(I) = \mu_2^0 + A_2{ I^2}^{ \alpha_1} =  \frac{(\eta_2 \dot{\gamma}^2 + \kappa_2)}{P},\
\label{eq_muI2}
\end{equation}
Then, we obtain $\mu_2 (\dot{\gamma}, P)$ and $\kappa_2(P)$
\begin{equation}
    \mu_2  (\dot{\gamma}, P)= 
    \biggl( A_2 {\left(   d  \sqrt{\rho} \right) }^{ 2\alpha_2}\biggr) 
     {\dot{\gamma}}^{2(\alpha_2-1)} P^{1-\alpha_2}
\label{eq_eta2}
\end{equation}
\begin{equation}
    \kappa_2(P) = \mu_2^0 P
\label{eq_kappa2}
\end{equation}

%--------------------------------------------------
We can obtain the flow fitting parameters from particle-based simulations, such as the discrete element method (DEM). The following parameter values are introduced in~\cite{jop_constitutive_2006} and~\cite{srivastava_viscometric_2021}:
    \[
    0.09 \leq \mu_1^0 \leq 0.33, 
    \ \ \ \ \ \ \ 
    0.37 \leq \alpha_1 \leq 0.7
    \]
        \[
    0.01 \leq \mu_2^0 \leq 0.1, 
    \ \ \ \ \ \ \ 
    0.28 \leq \alpha_2 \leq 0.44.
    \]
            \[
    \ \ \ \ \ \ \ 
    0.75 \leq \alpha_3 \leq 0.85.
    \]
     One can find the corresponding $A_i$ values in~\cite{srivastava_viscometric_2021}.
\begin{comment}
\section{Suspension rheology}

\end{comment}
\section{Numerical method}
In this section, we demonstrate the numerical method to solve the flow velocity with incompressible Navier-Stokes equations incorporating the various viscosities. We particularly focus on the time integration that we wanted to improve as we consider the second-order strain rate stress tensor.
For the rest of the computation, we use the AMReX-incflo library as it is.
\subsection{Velocity computation}
The main AMReX-incflo library provides three integrations: 1) explicit Euler, 2) Crank-Nicolson, and 3) implicit Euler methods. 
Since we have the stress tensor ${\bm \tau}$ of the form~(\ref{eq_CN_tau}), which is non-linear in the velocity, the most convenient choice was the explicit Euler method.
 However, it would be beneficial to have another scheme, for more stability. efficiency, and accuracy. 
 \par
 We consider a two-stage Runge-Kutta (RK2) method to achieve second-order convergence. 
In particular, we implemented the following scheme,
\begin{equation}
	\vec{u}_{j}^{temp} = \vec{u}^{n} + \frac{\Delta t}{2} {\bm F} \left( \vec{u}^{n} \right)
	\label{eq_RK2_s1} 
\end{equation}
\begin{equation}
	\vec{u}^{*} = \vec{u}^{n} + \Delta t {\bm F} \left( \vec{u}^{temp} \right),
	\label{eq_RK2_s2}
\end{equation}
where 
\[
  {\bm F} \left( \vec{u}^{n} \right)= 
    -\vec{u}^n \cdot \nabla \vec{u}^n 
    +\frac{1}{\rho}
    \biggl(
    - \nabla P^n 
        + \nabla \cdot   \bm{\tau}(\vec{u}^n)
        +  \rho  \vec{g} 
        \biggr).
\]
where the superscription $n$ represents the current time (known) value; $\vec{u}^n = \vec{u}(\vec{x}, t_n)$. The new velocity value $\vec{u}^*$ is an updated velocity to the next time, $\vec{u}^*  = \vec{u}^* (\vec{x}, t_{n+1})$. However, $\vec{u}^*$ does not necessarily satisfy the continuity equation~(\ref{eq_conserv_mass}). Thus, we project this intermediate velocity $\vec{u}^*$ onto the divergence-free space. 
\par
For the projection step, we express $\vec{u}^{*}$ according to the Helmholtz-Hodge-Decomposition~\cite{chorin_mathematical_1993}, 
\begin{equation}
  \vec{u}^* = \vec{u}^{n+1} + \nabla \phi,
  \label{eq_ustar}
\end{equation}
where $\vec{u}^{n+1}$ is the new velocity at time $t_n + \Delta t$ that satisfies the divergence-free condition and $\phi$ is a scalar function.
To take advantage of the incompressibility of $\vec{u}^{n+1}$, we find the divergence of equation~(\ref{eq_ustar}), leading to the Poisson equation for $\phi$,
\[
  \nabla^2 \phi = \nabla \vec{u}^*.  
\]
After we solve for $\phi$, applying given boundary conditions, we find the updated velocity, projecting the velocity $\vec{u}^*$ onto the divergence-free space to get the new time-level solutions.
\[
  \vec{u}^{n+1} = \vec{u}^* - \nabla \phi.
\]

% {\color{blue} Are we keeping this the whole time? Is there any pressure change over time? Does a volume fraction come into play?}

% a little more details
%---subsection---what I did not do-------
\subsection{Hydrostatic pressure dependency}
For granular rheology, it is typical to see compressible flow. It is, thus, essential to take pressure into account to evaluate the viscosity.
Here, we consider the pressure as a combination of background flow pressure, $P_0$, that is linear in the vertical direction, $z$, and perturbation, $P'$, such that
\[
P = P_0(z) + P'(x,y,z).\]
Note that the perturbations are due to gravity. Without a gravitational force, we may simply input the constant pressure, $P_0$, and keep it over time.
\par
In case gravity is involved, we consider the density term in order to approximate the perturbation, as 
\[
\rho  = \rho_0  + \rho'(x,y,z), 
\]
where $\rho_0$ is constant and $\rho'$ is a spacial-dependant density perturbation of the flow. 
Here, we assume that 
\begin{equation}
    \nabla P_0  = \frac{d P_0}{d z} \approx \rho_0  g.  
\label{eq_p0_rho0}
\end{equation}
This recovers our momentum equation. As we would like to construct a background pressure that stays constant over time for our granular rheology, we may use this assumption.
By integrating both sides of equation~(\ref{eq_p0_rho0}), we obtain
\begin{equation}
    P_0 \approx p_{bg} + \nabla P_0 z.
\end{equation}
We would like to use this form since we already have the $p_{bg}$ term implemented in the AMReX-incflo code.
%
When we have periodic boundary conditions in the gravity direction, we might need to prescribe a pressure gradient to have an additional pressure effect. 
\par
The challenge we faced in implementing the pressure-dependent viscosity was connecting the pressure in addition to the strain rate into the rheology code. 
We also need to use a different module than the AMReX-incflo for a compressible flow.
% We need to solve for a compressible flow model to have physical consistency using a different numerical solver rather than AMReX-incflo.
 We thus leave the pressure-dependent granular flow as a future work.
\begin{comment}
\subsection{Predictor-Corrector method}
To enlarge the numerical stability region, we implement a semi-implicit method. To do so, we first re-write the momentum equation by splitting the deviatoric stress tensor, ${\bm \tau}$, into two parts, 
\[
 {\bm \tau}= {\bm \tau_1} + {\bm \tau_2},
\] 
where ${\bm \tau_1}$ and ${\bm \tau_2}$ are, respectively, the linear and quadratic relationship with ${\bm D}$, i.e., 
\[
  {\boldsymbol \tau_1} \propto \boldsymbol{D}
  \ \ \ \ \ \text{and}
   \ \ \ \ \ 
{\boldsymbol \tau_2} \propto {\boldsymbol D^2}.
\]
We denote the advection term, $\vec{u} \cdot \nabla \vec{u}$ as ${\bm A}$. 
\\
$<${\bf Predictor}$>$
\\
First, consider the non-linear stress tensor, ${\bm \tau_2}$ as a force term, and solve for the first predictor values, that are $\vec{u}^*$ and $\vec{u}^*$:
\[
\frac{{\color{blue}\vec{u}^*} - \vec{u}^n}{\Delta t} 
+  {\bm A}^{n+1/2} 
= \frac{1}{\rho}  \biggl(
\frac{\nabla \cdot {\color{blue}{\bm \tau}_1^*} + \nabla \cdot{\bm \tau}_1^n}{2} 
+ \nabla \cdot {\bm \tau}_2^n 
- \nabla p^n
+{\bm g}
\biggr)
\]
We solve for blue terms - 
\[
{\color{blue}\vec{u}^*} -
\frac{\Delta t}{\rho} 
\left( 
\frac{\nabla \cdot {\color{blue}{\bm \tau}_1^*}}{2}
\right)
=
\vec{u}^n
- {\bm A}^{n+1/2} \Delta t
+ \frac{\Delta t}{\rho} \biggl(
\frac{ \nabla \cdot{\bm \tau}_1^n}{2} 
+ \nabla \cdot {\bm \tau}_2^n 
- \nabla p^n
+{\bm g}
\biggr)
\]
\\
$<${\bf Corrector}$>$
\\
Once we obtain the predictor (star) velocity, we use it to compute the second order stress tensor, ${\bm \tau}_2^*$.
\[
\frac{{\color{red}\vec{u}^{n+1}} - \vec{u}^n}{\Delta t} 
+  {\bm A}^{n+1/2} 
=  \frac{1}{\rho}  \biggl(
\frac{\nabla \cdot {\color{red}{\bm \tau}_1^{n+1}} + \nabla \cdot{\bm \tau}_1^n}{2} 
+ \frac{\nabla \cdot {\bm \tau}_2^{n} + \nabla \cdot {\color{blue}{\bm \tau}_2^*}}{2} 
- \nabla p^n
\biggr)
\]

Now, we need to solve for red - next time step values:
\[
{\color{red}\vec{u}^{n+1}} 
-\frac{1}{\rho} 
\left(
\frac{\nabla \cdot {\color{red}{\bm \tau}_1^{n+1}}}{2}
\right)
=
\vec{u}^n 
 -{\bm A}^{n+1/2} \Delta t
 +\frac{\Delta t}{\rho}  \biggl(
\frac{ \nabla \cdot{\bm \tau}_1^n}{2} 
+ \frac{\nabla \cdot {\bm \tau}_2^{n} + \nabla \cdot {\color{blue}{\bm \tau}_2^*}}{2} 
- \nabla p^n
\biggr)
\]
Note that ${\bm \tau}_i^* = {\bm \tau}(\vec{u}^*, \eta_i)$.
\end{comment}
\section{Numerical validation for time integrals}
We mainly compare the performance of two time integration methods; 1) the explicit Euler, which was already available in the AMReX-incflo, and 2) the explicit RK2 scheme that I implemented for granular rheology.
\begin{itemize}
  \item We use $16 \times 32 \times 16$ grid points in a long-kind cube.
  \item We consider a simple shear flow where the top of $z$-axis is moving with a velocity 0.1 (unit) under gravity (how much)?  \item We look at a material (what kind? - like sand) assuming bingham 
  \item Toothpaste?
  \item No-slip boundary conditions on the x and z axis AND (a large) periodic boundary condition is applied for the y direction (flow moving)
\end{itemize}



\subsection{Stability analysis}
\begin{figure}[ht]
  \begin{center}
    \includegraphics[scale=0.7]{figures/fig_stability_RK_Euler.pdf}
    \end{center}
  \caption{Stability regions of the explicit Euler and RK2 methods. }
  \label{fig_stability_RK_Euler}
\end{figure}
\begin{itemize}
  \item Show and compare the stability region
  \item Compute the eigenvalues of the problem using a finite difference scheme (first order)
  \item Show it is slightly better for this case since it is an advection dominant problem
\end{itemize}
\subsection{Error analysis (order convergence)}
\begin{itemize}
  \item Order convergence plot with various time step size 
\end{itemize}

\section{Discussion and future work}
There are multiple avenues we can go for future work.  
We may add the time-dependent flow presented in the last term in the following equation
\begin{equation}
  \bm{\tau} =  \mu_1 {\bm D} 
    + \mu_2  \left[ {\bm D}^2  - \frac{\text{tr}\left({\bm D}^2\right)}{3}{\bm I} \right]
   + \kappa_1 \frac{{\bm D}}{|{\bm D}|} 
    + \kappa_2  \left[ \frac{{\bm D}^2}{|{\bm D}|^2}  
    - \frac{\text{tr}\left({\bm D}^2\right)}{3|{\bm D}|^2}{\bm I} \right]
    + \mu_3  \left[ {\bm D}{\bm W} - {\bm W}{\bm D} \right]
  \end{equation}
  For the last coefficient in equation~(\ref{eq_2ndOrder_tau}), we follow the power law, 
\begin{equation}
    \mu_3(I) = -A_3 \left( I^2 \right)^{\alpha_3} = \frac{\eta_3 \dot{\gamma}^2}{p}.
\label{eq_muI3}
\end{equation}
Specifically, we can find the coefficients of each shear rate term in equation~(\ref{eq_2ndOrder_tau}) as
\begin{equation}
     \eta_3 (\dot{\gamma}, p) = 
    -p A_3 
        \left( \frac{\dot{\gamma} a }{\sqrt{p/\rho}}  \right)^{2\alpha_3} 
        \frac{1}{\dot{\gamma}^2},
\label{eq_gr_eta_3}
\end{equation}
\par
We propose to extend this work by incorporating elastic effects through the implementation of elastoviscoplastic (EVP) models in this numerical framework. The robustness of the numerical implementation will be extensively tested in various flow scenarios (such as Poiseuille and Couette flows) for a range of Weissenberg and Bingham numbers.  Another potential avenue for development will involve implementing immersed boundary methods (IBM) to model a suspension of solid particles in such complex fluids, which is an important application area that has received significant research interest lately. 
\par
Add surface tension effect in AMReX-incflo. 
\par
This section summarizes the rheological model of suspensions in a Newtonian fluid. Solid particles suspended in a Newtonian fluid can exhibit highly non-Newtonian behaviors.  
\par
We use the constitutive equation proposed by Reiner \cite{reiner_mathematical_1945} and Rivlin \cite{rivlin_stress-deformation_1955},  
\begin{equation}
  \bar{\bar{\sigma}} = -P \bar{\bar{I}}
  + 2 \nu_1 {\bm{D}} + 2 \nu_2 {\bm{D}}^2.
\end{equation}
According to Tanner's review \cite{tanner_review_2018}, the coefficients $\nu_1$ can be constant since the rate of change with $\dot{\gamma}$ is negligible, and $\nu_2 = \beta / \dot{\gamma}$ for a suspension of solid particles in a Newtonian fluid, where a factor $\beta$ is related to the strain rate $\dot{\gamma}$. Dai et, al \cite{dai_viscometric_2013} shows the $\beta = -4.4 \phi^3 \nu_1$ in shear flows, where $\phi$ is a volume fraction. 
\par
For numerical validations, we follow the experiments presented in Couturier $\textit{et, al}$~\cite{couturier_suspensions_2011}.
\begin{itemize}
  \item We want to observe the relationship between $\alpha_i(\phi) = N_i / |\sigma_{xy}|$, where $i = 1,2$.
\end{itemize}